\subsubsection*{Первая часть. Порождение}

В игре есть несколько объектов\+: \hyperlink{classCWarrior}{C\+Warrior}, \hyperlink{classCWorkers}{C\+Workers}, \hyperlink{classEnemyHeart}{Enemy\+Heart}. Воины и рабочие создаются с помощью пораждающего паттерна Фабричный Метод. Данный паттерн уместен здесь потому что у нас и воины и рабочие имеют схожие параметры и наследуются от базового класса \hyperlink{classUnits}{Units}. Поэтому реализуется общий интерфейс. Так же я оставил возможность расширять возможности. На данные момент существуют только воины ближнего боя, но в планах добавить так же воинов дальнего боя, как будет время.

Для создания \hyperlink{classEnemyHeart}{Enemy\+Heart} я использовал паттерн Синглтон. Он необходим потому что данный объект должен быть един для всех, то есть существовать в единственном экзампляре. Синглтон как раз предостеригает нас от создания копий данного объекта.

Так же я применил Синглтон для создания \hyperlink{classCMap}{C\+Map}. По той же самой причине, что карта должна существовать в единственном экзампляре. Плюс возможность взаимодействовать с картой из любого уголка кода, что тоже необходимо

\subsubsection*{Вторая часть. Структурирование}

Из структурных паттернов я использовал лишь Компоновщик. Он удобен для объединений воинов в \hyperlink{classCARMY}{C\+A\+R\+MY}. Армия представляет собой дерево, хоть это и дерево высоты 2, потому что не существует никаких мини группировок, таких как отряд, полк и тд. Просто воины собираются в Армию. Один воин уже может являться Армией. Создание Армии добавляет удобства в игру. Обединив воинов в армию, можно одновременно передвигать сразу всех юнитов, а так же атаковать сразу всеми юнитами. Паттерн реализован так, что если необходимо нескольми юнитами из армии сделать какое-\/либо действие, отличающееся от действия, которое должно сделать вся Армия, то не нужно расформировывать всю Армию. Достаточно лишь удалить воина из армии, после чего все его действия станут индивидуальными.