\subsubsection*{Первая часть. Порождение}

В игре есть несколько объектов\+: \hyperlink{classCWarrior}{C\+Warrior}, \hyperlink{classCWorkers}{C\+Workers}, \hyperlink{classEnemyHeart}{Enemy\+Heart}. Воины и рабочие создаются с помощью пораждающего паттерна Фабричный Метод. Данный паттерн уместен здесь потому что у нас и воины и рабочие имеют схожие параметры и наследуются от базового класса \hyperlink{classUnits}{Units}. Поэтому реализуется общий интерфейс. Так же я оставил возможность расширять возможности. На данные момент существуют только воины ближнего боя, но в планах добавить так же воинов дальнего боя, как будет время.

Для создания \hyperlink{classEnemyHeart}{Enemy\+Heart} я использовал паттерн Синглтон. Он необходим потому что данный объект должен быть един для всех, то есть существовать в единственном экзампляре. Синглтон как раз предостеригает нас от создания копий данного объекта.

Так же я применил Синглтон для создания \hyperlink{classCMap}{C\+Map}. По той же самой причине, что карта должна существовать в единственном экзампляре. Плюс возможность взаимодействовать с картой из любого уголка кода, что тоже необходимо

\subsubsection*{Вторая часть. Структурирование}

Из структурных паттернов я использовал лишь Компоновщик. Он удобен для объединений воинов в \hyperlink{classCARMY}{C\+A\+R\+MY}. Армия представляет собой дерево, хоть это и дерево высоты 2, потому что не существует никаких мини группировок, таких как отряд, полк и тд. Просто воины собираются в Армию. Один воин уже может являться Армией. Создание Армии добавляет удобства в игру. Обединив воинов в армию, можно одновременно передвигать сразу всех юнитов, а так же атаковать сразу всеми юнитами. Паттерн реализован так, что если необходимо нескольми юнитами из армии сделать какое-\/либо действие, отличающееся от действия, которое должно сделать вся Армия, то не нужно расформировывать всю Армию. Достаточно лишь удалить воина из армии, после чего все его действия станут индивидуальными.

Для остальных паттернов я не придумал применения, в моей игре они не нужны. Нет объектов которые нужно чем-\/либо обернуть. Нет нужды применять адаптер, т.\+к. все необходимые объекты и так спокойно контактируют друг с другом. Для прокси тоже не вижу уместного применения.

\subsubsection*{Правила игры}

Изначально у игрока есть N монет, которые он может потратить либо на покупку воинов, либо на покупку рабочих. Рабочие приносят доход каждый ход, но военные показатели(здоровье, урон и тд) у них слабые, поэтому если до них доберутся, то почти сразу убьют. С другой стороны карты находится Сердце противника, здание, которое каждые k ходов пораждает около 4 воинов. Войны идут в наступление, их цель дойти до левого края карты. Цель игры уничтожить \char`\"{}Сердце противника\char`\"{} как можно быстрее. Но сделать это не так-\/то просто. У него есть свои показатели здоровья и брони. Так же оно наносит урон каждую секунду(Будем считать что скорость атаки у всех юнитов 1 удар в секунду). Поэтому если войны дошли до сердца, то зачастую они могут сделать не более 3 ударов. Воинов можно объединять в Армии для удобства управления. Армия может двигаться не нарушая взаимное расположение юнитов и атаковть цель, которая находится в радиусе атаки.